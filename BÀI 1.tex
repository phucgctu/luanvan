\documentclass[12pt,a4paper]{book}
\usepackage[utf8]{inputenc}
\usepackage[english,vietnam]{babel}
\usepackage{enumerate}
\usepackage{amsmath}
\usepackage[standard,thmmarks]{ntheorem}%thref,amsmath,hyperref
\theoremstyle{plain}%Sử dụng phong cách plain để đánh số Định lý, chẳng hạn: Định lý 1, Định lý 2,...
\theoremseparator {.}% Ngăn cách giữa Định lý và nội dung định lý bởi dấu chấm%
\renewtheorem{theorem}{Định lý}
\renewtheorem{proof}{Chứng minh}
\renewtheorem{definition}{Định nghĩa}
\renewtheorem{lemma}{Bổ đề}
\theoremheaderfont{\normalfont\bfseries}% In ra chữ "Định lý": đậm, thẳng đứng.%
\usepackage{amsfonts}
\usepackage{amssymb}
\usepackage[left=2cm,right=2cm,top=2cm,bottom=2cm]{geometry}
\author{Đào Duy Phúc - MSSHV: M0719005}
\title{HỌC TIN HỌC VỚI BẰNG PHẦN MỀN \LaTeX}
\usepackage{fancyhdr}  %Tạo header và Footer
\pagestyle{fancy}
\lhead{\textbf{Đại học cần thơ}}
\chead{\textit{Latex}}
\rhead{\textit{Toán giải tích}}
\lfoot{Đào Duy Phúc}
\cfoot{\thepage} %đánh số trang%
\setlength{\parindent}{0pt} % Tất cả các đoạn văn bản đề thục ra ngoài% 
\begin{document}
\selectlanguage{vietnam}
\maketitle %Hiện tiêu đề%
\chapter[chương 1:] {Phép tính vi tích phân hàm một biến}
\section{khái niệm về đạo hàm}
\section{Các phương pháp tính đạo hàm}
\section{Vi phân}
\fontsize{17pt}{40pt}\selectfont
\textbf{Câu 2: Tạo các biểu thức}
\fontsize{17pt}{40pt}\selectfont\\
$ x\vee y\wedge z, \qquad x\cup y\cap z, \qquad x\circ y, \qquad A\subset B\subseteq C,\qquad A\backslash B $\\
\textbf{Câu 3: Tạo ra đẳng thức sau:}$\qquad\displaystyle \sum_{x\in A}f(x)\stackrel{de f}{=}\displaystyle
\sum _{x\in A\atop x\neq 0}f(x)$\\
\textbf{Câu 4: Tạo ra mệnh đề}\\
\begin{center}
$\dfrac{f(x+\bigtriangleup x)-f(x)}{\bigtriangleup x}\longrightarrow f'(x)$ \mbox{khi} $ \bigtriangleup x\rightarrow0$\\
\end{center}
\textbf{Câu 5: Tạo ra các tích phân:}$\qquad\int\limits_a^b {f(x)dx = \left. {F\left( x \right)} \right|_a^b} $\\
\textbf{Câu 6:} Tạo ra dạng: \qquad$\left\| {{u_i}} \right\| = 1,$ $\qquad{u_i}.{u_j} = 0$\qquad \mbox{nếu}\qquad $i \ne j$
\textbf{Câu 7:} Tạo ra ma trận sau:\\
\[\left[ {\begin{array}{*{20}{c}}
{{a_{11}}}&{{a_{12}}}&{...}&{{a_{1n}}}\\
{{a_{21}}}&{{a_{22}}}&{...}&{{a_{2n}}}\\
 \vdots & \vdots & \vdots & \vdots \\
{{a_{m1}}}&{{a_{m2}}}&{...}&{{a_{mn}}}
\end{array}} \right]\]
\textbf{Câu 8:} Tạo ra dạng sau:\\
\begin{eqnarray}
T\left( n \right) & \le & c\left( {{2^{\left[ {\lg n} \right]}} - {2^{\left[ {\lg n} \right]}}} \right)\nonumber\\
 &\le & 3c \cdot {3^{\lg n}} \nonumber\\
 & = & 3c{n^{\lg 3}}\nonumber\\
 \nonumber
\end{eqnarray}
Câu 9: Dùng môi trường equation tạo ra dạng sau:\\
\begin{equation}
{x( {t,\tau ,\xi ,\eta })=  \bar x( {\tau  - t;\tau ,\xi ,\eta } ),
\atop
y( {t,\tau ,\xi ,\eta } ) = \bar  y( {\tau  - t;\tau ,\xi ,\eta } )},
\qquad\qquad(\tau,\xi,\eta)\tag{6.1}
\end{equation}
\textbf{câu 10:} Dùng môi trường minipage tạo ra dạng sau:\\
\begin{minipage}{8cm}
Cho cung đường tròn $AB$ có phương trình $y = f\left( x \right)$  $\left( {a \le x \le b} \right)$, trong đó $f\left( x \right)$ có đạo hàm liên tục trên $\left[ {a,b} \right]$.\\
Mặt tròn xoay tạo nên khi quay cung $AB$ quanh trục $Ox$ có diện tích cho bởi
\end{minipage}
\begin{minipage}[t]{5.2cm}
\fbox{$S = 2\pi \int\limits_a^b {\left| {f\left( x \right)} \right|} \sqrt {1 + f'\left( x \right)} dx$}
\end{minipage}
\vspace{6pt}\underline{\textbf{Câu 10}}: Dùng môi trường minipage tạo ra dạng sau:\\

\begin{minipage}{8cm}
Cho cung đường tròn AB có phương trình $y=f(x) \hspace{6pt}(a\le x\le b),$ trong đó $f(x)$ có đạo hàm liên tục trên $[a,b].$\\
Mặt tròn xoay tạo nên khi quay cung $AB$ quanh trục $Ox$ có diện tích cho bởi
\end{minipage}
\qquad\begin{minipage}[t]{5cm}
\parbox{8cm}{
\fbox{$S = 2\pi \int\limits_a^b {\left| {f\left( x \right)} \right|\sqrt {1 + f{'^2}} dx} $}
}
\end{minipage}
\end{document}
